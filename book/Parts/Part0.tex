\clearpage
\section*{Foreword}\index{Foreword}
This book evolved out of my lecture notes for the undergraduate and graduate computational fluid dynamics courses that I teach at Concordia University. In combination with an open-educational resources grant from the library, it is provided to you completely free of charge. It uses entirely open-source tools including Python, Jupyter Notebooks, SU2, Gmsh, Paraview, \LaTeX, and many others. This means that all of the examples and applications can be run on your computer using any common operating system and without purchasing any licenses - CFD for free! 

The book itself is even open-source, available as a public repository on Gitlab.  As such, you may want to think of this book more as a software development project, rather than a conventional hard cover textbook. All of the content needed to run the chapter examples and final applications are stored in the repository, and you can even contribute to the book and its examples via a pull request if you think of a useful addition. Just like any software development project, this book may contain a few ``bugs'', mostly in the form of minor typos. If you find one of these please feel free to give back and submit a pull request to correct them.

In terms of content, the book is designed with enough material to cover an advanced undergraduate course, or an introductory course for graduate students. For an undergraduate course I recommend a more hands-on computer lab experience, focusing on Part 1, some of Part 2, and Part 3. In Part 2 I find it useful to cover finite difference methods, consistency, stability, convergence,  time stepping, iterative methods, and then return to introduce finite volume methods. Part 3 is designed as a bi-weekly computer lab, where students get hands-on experience with practical CFD simulations. For a graduate course I recommend focusing on Parts 1 and 2, with a final project to write a two-dimensional compressible solver for lid driven cavity flow. If you are studying CFD on your own then I recommend covering the whole book.

Finally, I would like to thank my undergraduate and graduate students over the years for their useful discussions and contributions to the core ideas in this book. In particular, I need to acknowledge my co-authors Carlos and Hamid for their hard work in developing the first version. I also thank the students in my classes for using the first ``experimental'' editions and their useful feedback. Finally, I would like to thank you the reader for your interest in this project and for learning CFD.

\vspace{0.5cm}

\noindent Dr. Brian C. Vermeire