\part{Part 3: Numerics}

\chapterimage{chapter_head_2.pdf} % Chapter heading image

\chapter{Taylor-Series}
To begin our exploration of CFD we will state with the Finite Difference Method (FDM). This approach uses conservation laws in {\it divergence form}. We start by recalling a few of the simplified conservation laws we derived in the previous section, specifically the linear advection, Burgers, and linear diffusion equations
\begin{eqBox}
\begin{equation}
	\frac{\partial u}{\partial t} +  \alpha \frac{\partial u}{\partial x} = 0,
\end{equation}
\begin{equation}
	\frac{\partial u}{\partial t} +  \frac{1}{2} \frac{\partial u^2}{\partial x} = 0,
\end{equation}
\begin{equation}
	\frac{\partial u}{\partial t} - \alpha \frac{\partial^2 u}{\partial x^2} = 0.
\end{equation}
\end{eqBox}
We notice that the form of all of these equations is very similar. All three involve a time derivative combined with a coefficient multiplied by a spatial derivative. The Finite Difference Method uses well-known concepts from applied mathematics, specifically Taylor-Series. Hence, before deriving the Finite Difference Method we will start by reviewing some concepts from Taylor-Series.

Arising from Taylors theorem, Taylor series says that the value of a sufficiently smooth function at some point $x + \Delta x$ can be predicted by using the value of the solution at the point $x$ along with knowledge of all derivatives at the point $x$.
\begin{theorem}[Taylor's Theorem]
Let $k \geq 1$ and letting $f(x)$ be smooth and differentiable $k$ times then
\begin{align}
f(x + \Delta x) = f(x) + \frac{\partial f}{\partial x}\frac{\Delta x^1}{1!} + \frac{\partial^2 f}{\partial x^2}\frac{\Delta x^2}{2!} + \hdots + \frac{\partial^k f}{\partial x^k}\frac{\Delta x^k}{k!},
\end{align}
which can be written compactly for an expansion truncated to n terms as
\begin{align}
f(x + \Delta x) = \sum_{i=0}^{k} \frac{\partial^i f}{\partial x^i}\frac{\Delta x^i}{i!}.
\end{align}
\end{theorem}
While there are some limitations, particularly around the smoothness of the function and its derivatives, it is worth taking a moment to consider just how powerful Taylor series can be. It allows us to represent the entirety of a smooth function using the value of the solution and its derivatives at only one point in the domain. In addition, similar to Fourier series, we often find that a truncated expansion, which omits high-order terms in the summation, is sufficient for many applications.

We will demonstrate the utility of Taylor series via example. Consider a simple sine wave 
\begin{equation}
	f(x) = \sin(x),
\end{equation}
which is periodic on the interval $[- \pi, \pi]$. We first note that sine functions are sufficiently smooth for Taylor series to be applied up to an infinite number of derivatives. Now we will explore the behaviour of the Taylor series as we add more terms. When we use a finite number of terms in the expansion we are {\it truncating} all of the higher-order terms. We will form our expansion about the point $x = 0$. If we include just the first term and truncate all others we obtain
\begin{equation}
	f_0(\Delta x) = 0,
\end{equation}
which is not a particularly accurate approximation of a sine wave except very close to the origin. However, as we start to add more terms we obtain the following expressions,
\begin{equation}
	f_1(\Delta x) = \Delta x,
\end{equation}
\begin{equation}
	f_3(\Delta x) = \Delta x - \frac{\Delta x^3}{6},
\end{equation}
\begin{equation}
	f_5(\Delta x) = \Delta x - \frac{\Delta x^3}{6} + \frac{\Delta x^5}{120},
\end{equation}
\begin{equation}
	f_7(\Delta x) = \Delta x - \frac{\Delta x^3}{6} + \frac{\Delta x^5}{120} - \frac{\Delta x^7}{5040},
\end{equation}
\begin{equation}
	f_{9}(\Delta x) = \Delta x - \frac{\Delta x^3}{6} + \frac{\Delta x^5}{120} - \frac{\Delta x^7}{5040} + \frac{\Delta x^9}{362880},
\end{equation}
where the subscript denotes the power of the highest degree expansion term included in the approximation. Each of these is a polynomial approximation of a sine wave about the point $x=0$. We can make a few observations about the behaviour of Taylor series in general. First, when we are close to the expansion point $x=0$, even approximations with a small number of terms are close to the true function. For example, the $f_1$ expansion is the well-known small angle approximation of a sine wave. Then, as we add more terms the accuracy of the expansion improves rapidly. For example, by ten terms the approximation is nearly indistinguishable from the true function on the entire interval. Hence, we note that Taylor series becomes more accurate as we get closer to the expansion point and/or as we increase the number of terms.

To understand the behaviour of these truncated expansions, we note that their error arises from truncating the higher-order terms of the expansion. Furthermore, when $\Delta x$ is relatively small the error is dominated by the first truncated term, since the higher-order terms rapidly approximate zero as $\Delta x$ decreases. Hence, if we consider an infinite expansion
\begin{equation}
	f(\Delta x) = \Delta x - \frac{\Delta x^3}{6} + \frac{\Delta x^5}{120} - \frac{\Delta x^7}{5040} + \frac{\Delta x^9}{362880} - \frac{\Delta x^{11}}{39916800} + \hdots,
\end{equation}
we note that the leading term omitted from the $f_1(\Delta x)$ approximation behaves like $\Delta x^3$, the leading error term of the $f_1(\Delta x)$ approximation behaves like $\Delta x^5$, and so on. Hence, as we get closer to the expansion point $x=0$ we expect that the error shrinks, and that the rate at which the error shrinks will be proportional to the exponent of the leading truncated term of the expansion. We can test this by creating a convergence table.
	
\chapter{Finite Difference Methods}
\section{The First Derivative}
To introduce the finite difference method we start with a general function $u(x)$. In general this function can be in any number of dimensions, but for simplicity we will first consider the one-dimensional case. We note that a function can generally be {\it approximated} by its values at a {\it discrete} set of points. An example of this is shown in Figure \ref{} where $\Delta x$ is the grid spacing between the points. We now imagine that we are at some point $i$ whose solution is $u_i = u(x_i)$ where $x_i$ is the physical location of the point. Using Taylor series we can compute the value of the solution one grid point to the left from the expansion
\begin{equation}
	u_{i-1} = u_i - \Delta x \Eval{\frac{\partial u}{\partial x}}{x_i}{} + \frac{\Delta x^2}{2} \Eval{\frac{\partial^2 u}{\partial x^2}}{x_i}{} - \frac{\Delta x^3}{6} \Eval{\frac{\partial^3 u}{\partial x^3}}{x_i}{} + \mathcal{O}(\Delta x^4),
	\label{eqn:fdmup}
\end{equation}
where $\mathcal{O}(\Delta x^4)$ denotes the order of the next highest term in the expansion. We can also use another Taylor series to predict the value of the solution one point to the right
\begin{equation}
	u_{i+1} = u_i + \Delta x \Eval{\frac{\partial u}{\partial x}}{x_i}{} + \frac{\Delta x^2}{2} \Eval{\frac{\partial^2 u}{\partial x^2}}{x_i}{} + \frac{\Delta x^3}{6} \Eval{\frac{\partial^3 u}{\partial x^3}}{x_i}{} + \mathcal{O}(\Delta x^4).
	\label{eqn:fdmdwn}
\end{equation}

So far, we have just applied Taylor series to try to determine the solution at points around our initial point $u_i$. However, if we look at Equation \ref{eqn:fdmup} we can use it in a different way. Rearranging we get
\begin{equation}
	\Eval{\frac{\partial u}{\partial x}}{x_i}{} = \frac{u_i - u_{i-1}}{\Delta x} + \frac{\Delta x}{2} \Eval{\frac{\partial^2 u}{\partial x^2}}{x_i}{} - \frac{\Delta x^2}{6} \Eval{\frac{\partial^3 u}{\partial x^3}}{x_i}{} + \mathcal{O}(\Delta x^3).
	\label{eqn:fdmup1}
\end{equation}
We note that as the grid spacing gets smaller the terms with leading $\Delta x$, $\Delta x^2$, and so on will tend to zero and the approximation
\begin{eqBox}
\begin{equation}
	\Eval{\frac{\partial u}{\partial x}}{x_i}{} = \frac{u_i - u_{i-1}}{\Delta x} + \mathcal{O}(\Delta x),
\end{equation}
\end{eqBox}
will converge to the true value of the derivative at the point $x_i$ as $\Delta x$ tends to zero. We can also do the same trick with Equation \ref{eqn:fdmdwn}. Rearranging we get
\begin{equation}
	\Eval{\frac{\partial u}{\partial x}}{x_i}{} = \frac{u_{i+1} - u_{i}}{\Delta x} - \frac{\Delta x}{2} \Eval{\frac{\partial^2 u}{\partial x^2}}{x_i}{} - \frac{\Delta x^2}{6} \Eval{\frac{\partial^3 u}{\partial x^3}}{x_i}{} + \mathcal{O}(\Delta x^3),
	\label{eqn:fdmdwn1}
\end{equation}
and again if the grid spacing is small then the higher-order terms tend to zero and we are left with
\begin{eqBox}
\begin{equation}
	\Eval{\frac{\partial u}{\partial x}}{x_i}{} = \frac{u_{i+1} - u_{i}}{\Delta x} + \mathcal{O}(\Delta x),
\end{equation}
\end{eqBox}
which also allows us to approximate the derivative at $x_i$. Hence, if we know that value of the solution at $x_i$ and either $x_{i-1}$ or $x_{i+1}$ we can approximate the derivative at $x_i$. We note that for both of these approximations the error in the derivative approximation will decrease proportional to the grid spacing, since the leading term in the truncation error is of $\mathcal{O}(\Delta x)$. We refer to a scheme of this form as a {\it first-order} finite difference method.

While the above examples are useful they converge rather slowly to the exact value of the derivative at the point $x_i$. However, if we take an average of Equation \ref{eqn:fdmup1} and Equation \ref{eqn:fdmdwn1} we obtain
\begin{equation}
	\begin{aligned}
	\Eval{\frac{\partial u}{\partial x}}{x_i}{} = &\frac{1}{2}\left(\frac{u_i - u_{i-1}}{\Delta x} + \frac{\Delta x}{2} \Eval{\frac{\partial^2 u}{\partial x^2}}{x_i}{} - \frac{\Delta x^2}{6} \Eval{\frac{\partial^3 u}{\partial x^3}}{x_i}{} \right) + \\
	& \frac{1}{2}\left(\frac{u_{i+1} - u_{i}}{\Delta x} - \frac{\Delta x}{2} \Eval{\frac{\partial^2 u}{\partial x^2}}{x_i}{} - \frac{\Delta x^2}{6} \Eval{\frac{\partial^3 u}{\partial x^3}}{x_i}{} \right) + \mathcal{O}(\Delta x^3)
	\end{aligned}
\end{equation}
which simplifies to
\begin{equation}
	\Eval{\frac{\partial u}{\partial x}}{x_i}{} = \frac{u_{i+1} - u_{i-1}}{2 \Delta x} - \frac{\Delta x^2}{6} \Eval{\frac{\partial^3 u}{\partial x^3}}{x_i}{} + \mathcal{O}(\Delta x^3),
\end{equation}
noting that the leading truncation error terms conveniently cancelled each other out. Hence, if we assume that $\Delta x$ is relatively small we obtain
\begin{eqBox}
\begin{equation}
	\Eval{\frac{\partial u}{\partial x}}{x_i}{} = \frac{u_{i+1} - u_{i-1}}{2 \Delta x} + \mathcal{O}(\Delta x^2).
\end{equation}
\end{eqBox}
Using this equation we can approximate the value of the derivative of the function at $x_i$ using the values of the solution at two neighbouring points $x_{i-1}$ and $x_{i+1}$. However, unlike the previous finite difference methods, the truncation error of this approximation is $\mathcal{O}(\Delta x^2)$, meaning that as the grid spacing is reduced the error will converge like the grid spacing squared. We refer to a scheme of this form as a {\it second-order} finite difference method.

\section{A General Approach}
In the previous section we derived three different finite difference methods for computing the derivative at a point $x_i$ using the values of the solution at that point and neighbouring points. We have shown that we can create at least two first-order schemes and one second-order scheme. In the current section we will demonstrate how to generalize this approach to allow us to obtain schemes of any order and for higher-order derivatives.

To start with, we will derive a second-order finite difference method to approximate the first derivative using $u_i$, $u_{i-1}$, and $u_{i-2}$. This procedure can be broken down into four basic steps.

\subsection{Step 1: Generate the Taylor Series}
The first step is to expand a Taylor series about each of the points that will be used to approximate the derivative. The Taylor series for $u_{i-1}$ was given before and is
\begin{equation}
	u_{i-1} = u_i - \Delta x \Eval{\frac{\partial u}{\partial x}}{x_i}{} + \frac{\Delta x^2}{2} \Eval{\frac{\partial^2 u}{\partial x^2}}{x_i}{} - \frac{\Delta x^3}{6} \Eval{\frac{\partial^3 u}{\partial x^3}}{x_i}{} + \mathcal{O}(\Delta x^4).
	\label{fdm:general1}
\end{equation}
Similarly, since the distance between $u_i$ and $u_{i-2}$ is $2 \Delta x$ we get the following Taylor series for that point
\begin{equation}
	u_{i-2} = u_i - 2 \Delta x \Eval{\frac{\partial u}{\partial x}}{x_i}{} + \frac{(2\Delta x)^2}{2} \Eval{\frac{\partial^2 u}{\partial x^2}}{x_i}{} - \frac{(2\Delta x)^3}{6} \Eval{\frac{\partial^3 u}{\partial x^3}}{x_i}{} + \mathcal{O}(\Delta x^4).
	\label{fdm:general2}
\end{equation}

\subsection{Step 2: Rearrange the Taylor Series}
The second step is to rearrange each of the Taylor series generated in Step 1 to obtain the derivative of interest on the LHS with everything else on the RHS. Since we are interested in finding an approximation for the first derivative we rearrange Equation \ref{fdm:general1}
\begin{equation}
	\Eval{\frac{\partial u}{\partial x}}{x_i}{} = \frac{u_i - u_{i-1}}{\Delta x} + \frac{\Delta x}{2}\Eval{\frac{\partial^2 u}{\partial x^2}}{x_i}{} - \frac{\Delta x^2}{6}\Eval{\frac{\partial^3 u}{\partial x^3}}{x_i}{} + \mathcal{O}(\Delta x^3),
	\label{eqn:aHK8G4M7}
\end{equation}
and rearrange Equation \ref{fdm:general1}
\begin{equation}
	\Eval{\frac{\partial u}{\partial x}}{x_i}{} = \frac{u_i - u_{i-2}}{2\Delta x} + \frac{2\Delta x}{2}\Eval{\frac{\partial^2 u}{\partial x^2}}{x_i}{} - \frac{(2\Delta x)^2}{6}\Eval{\frac{\partial^3 u}{\partial x^3}}{x_i}{} + \mathcal{O}(\Delta x^3),
	\label{eqn:6sbrhgAM}
\end{equation}
which gives two expressions for the first derivative.

\subsection{Step 3: Determine a Suitable Combination}
If we look at the previous two expressions for the first derivative we will notice that they are both first-order accurate, since the leading term that will get truncated is $\mathcal{O}(\Delta x)$. However, we were tasked with finding a second-order accurate scheme. In order to achieve this we will try to combine Equation \ref{eqn:aHK8G4M7} and Equation \ref{eqn:6sbrhgAM} in such a way the this first-order error term cancels out.

We want to combine them is such a way that
\begin{equation}
	\Eval{\frac{\partial u}{\partial x}}{x_i}{} = a(\ref{eqn:aHK8G4M7}) + b(\ref{eqn:6sbrhgAM})
	\label{eqn:2KGWrZa5}
\end{equation}
gives a second-order approximating for the first derivative, and we need to find the coefficients $a$ and $b$ to achieve this. If we look at the left hand side, we want to keep the first derivative there. This can be achieve if we ensure that
\begin{equation}
	a + b = 1.
	\label{eqn:3Wxy2aCD}
\end{equation}
The second thing we need to do is cancel out the $\mathcal{O}(\Delta x)$ term to ensure that the leading truncation error term is $\mathcal{O}(\Delta x^2)$. Looking at the $\mathcal{O}(\Delta x)$ terms, in order for them to cancel out we require
\begin{equation}
	\frac{a}{2} + b = 0,
	\label{eqn:rWD4Hs8v}
\end{equation}
to cancel them out. Equations \ref{eqn:3Wxy2aCD} and \ref{eqn:rWD4Hs8v} are recognizable as a linear system of equations with two equations and two unknowns. This can be readily solved using substitution yielding $a=2$ and $b=-1$.

\subsection{Step 3: Combine the Schemes}
Now that we have determine the constants in Equation \ref{eqn:2KGWrZa5}, the last step is to just go ahead and add things together. Doing this will yield that following finite difference approximation
\begin{eqBox}
\begin{equation}
	\Eval{\frac{\partial u}{\partial x}}{x_i}{} = \frac{3u_i - 4u_{i-1} + u_{i-2}}{2 \Delta x} + \mathcal{O}(\Delta x^2),
	\label{eqn:aHK8G4M7}
\end{equation}
\end{eqBox}
which is a second-order approximation of the first derivative use three points. Hence, we have accomplished our task.

\section{The Second Derivative}
In the previous sections we derived four difference schemes for the first derivative, two that were first-order accurate, and two that were second-order accurate. However, if we go back and look at the Navier-Stokes equations, we will note that we also need to approximate second-derivatives for the diffusive terms. We will derive and example scheme here, which also demonstrates another application of the four basic steps of creating a finite difference scheme. Our objective will be to derive a second-order finite difference method to approximate the second derivative using $u_i$, $u_{i-1}$, and $u_{i+1}$.

We now recall that the first step was to simply expand a Taylor series about all of the points that are not $u_i$. We have already derived these expansions for $u_{i-1}$ and $u_{i+1}$, which are
\begin{equation}
	u_{i-1} = u_i - \Delta x \Eval{\frac{\partial u}{\partial x}}{x_i}{} + \frac{\Delta x^2}{2} \Eval{\frac{\partial^2 u}{\partial x^2}}{x_i}{} - \frac{\Delta x^3}{6} \Eval{\frac{\partial^3 u}{\partial x^3}}{x_i}{} + \mathcal{O}(\Delta x^4),
	\label{eqn:Bm6Tq8Ah}
\end{equation}
and
\begin{equation}
	u_{i+1} = u_i + \Delta x \Eval{\frac{\partial u}{\partial x}}{x_i}{} + \frac{\Delta x^2}{2} \Eval{\frac{\partial^2 u}{\partial x^2}}{x_i}{} + \frac{\Delta x^3}{6} \Eval{\frac{\partial^3 u}{\partial x^3}}{x_i}{} + \mathcal{O}(\Delta x^4).
	\label{eqn:93Nxbhtu}
\end{equation}
So that is step one done.

In step two we needed to rearrange these equations such that the derivative of interest is on the LHS. This yields two possible expressions for the second derivative by rearranging Equation \ref{eqn:Bm6Tq8Ah}
\begin{equation}
	\Eval{\frac{\partial^2 u}{\partial x^2}}{x_i}{} = \frac{2(u_{i-1} - u_i)}{\Delta x^2} + \frac{2}{\Delta x} \Eval{\frac{\partial u}{\partial x}}{x_i}{} + \frac{\Delta x}{3} \Eval{\frac{\partial^3 u}{\partial x^3}}{x_i}{} + \mathcal{O}(\Delta x^2),
\end{equation}
and Equation \ref{eqn:93Nxbhtu}
\begin{equation}
	\Eval{\frac{\partial^2 u}{\partial x^2}}{x_i}{} = \frac{2(u_{i+1} - u_i)}{\Delta x^2} - \frac{2}{\Delta x} \Eval{\frac{\partial u}{\partial x}}{x_i}{} - \frac{\Delta x}{3} \Eval{\frac{\partial^3 u}{\partial x^3}}{x_i}{} + \mathcal{O}(\Delta x^2),
\end{equation}
noting that the $\mathcal{O}(\Delta x^4)$ terms are now $\mathcal{O}(\Delta x^2)$ since we had to divide the RHS by $\Delta x^2$. That marks the end of step two.

In the third step we have to determine a linear combination of these two equations that will give us an expression for the second derivative with a truncation error of $\mathcal{O}(\Delta x^2)$. Since we want to keep the second derivative on the LHS we require
\begin{equation}
	a + b = 1.
\end{equation}
And, in order for the truncation error to be of $\mathcal{O}(\Delta x^2)$ we need to cancel the second and third terms on the RHS of the equations, which are $\mathcal{O}(\Delta x^{-1})$ and $\mathcal{O}(\Delta x)$, respectively. Interestingly, cancelling out both of these terms requires
\begin{equation}
	a - b = 0.
\end{equation}
Once again, we have a linear system with two equations and two unknowns. Solving this system yields $a=b=1/2$, which finishes part three.

Finally, in part four we simply combine the two equations multiplied by their respective $a$ and $b$ coefficients. This yields our second-order accurate expression for the second-derivative as
\begin{eqBox}
\begin{equation}
	\Eval{\frac{\partial^2 u}{\partial x^2}}{x_i}{} = \frac{u_{i-1} - 2u_i + u_{i+1}}{\Delta x^2} + \mathcal{O}(\Delta x^2).
\end{equation}
\end{eqBox}

\section{Example Applications}
Now that we have created a few introductory finite difference methods, it is time to put them to work on our three simplified systems.
\subsection{Linear Advection}
We recall that the one-dimensional linear advection equation in differential form was
\begin{equation}
	\frac{\partial u}{\partial t} +  \alpha \frac{\partial u}{\partial x} = 0.
\end{equation}
In the previous sections we derived a few options for this, but to start we will use the simple first-order upwind scheme to approximate the spatial derivative at each grid point, such that
\begin{equation}
	\Eval{\frac{\partial u}{\partial x}}{x_i}{} = \frac{u_i^t - u_{i-1}^t}{\Delta x} + \mathcal{O}(\Delta x),
\end{equation}
where the superscript $t$ denotes that we are using the solution at the current time. We will also use a first-order finite difference approach in time to approximate the temporal derivative as well. Hence,
\begin{equation}
	\Eval{\frac{\partial u}{\partial t}}{x_i}{} = \frac{u_i^{t+1} - u_{i}^t}{\Delta t} + \mathcal{O}(\Delta t),
\end{equation}
where $\Delta t$ is the time step size and the superscript $t+1$ is the solution at the next time step.

Now we simply insert these approximations into the linear advection equation, yielding
\begin{equation}
	\frac{u_i^{t+1} - u_{i}^t}{\Delta t} +  \alpha \frac{u_i^t - u_{i-1}^t}{\Delta x} + \mathcal{O}(\Delta x, \Delta t) = 0.
\end{equation}
By inspecting this equation, and assuming that we know the solution at each grid point at the current instant in time, we can rearrange this to make a prediction for the solution at the next time step
\begin{eqBox}
\begin{equation}
	u_i^{t+1} = u_{i}^t -\frac{\alpha \Delta t}{\Delta x} \left( u_i^t - u_{i-1}^t \right) + \mathcal{O}(\Delta x, \Delta t).
	\label{eqn:7hH8UBeG}
\end{equation}
\end{eqBox}

Looking at Equation \ref{eqn:7hH8UBeG} we note that, provided we know the current solution at all grid points at time $t$, we can predict the solution at time $t+\Delta t$ with an error term of $\mathcal{O}(\Delta x, \Delta t)$. Then we will have an approximation of the solution at all grid points at time $t+\Delta t$, and we can use this to find an approximation of the solution at time $t+2\Delta t$ by applying the same equation, and so on. Hence, we are able to advance the linear advection equation to any final time we desire, noting that each time step we take will introduce some numerical error.

\subsection{Burgers Equation}
We recall that the one-dimensional Burgers equation was
\begin{equation}
	\frac{\partial u}{\partial t} +  \frac{1}{2} \frac{\partial u^2}{\partial x} = 0.
\end{equation}
We can now use the exact same approach for Burgers as we used for the linear advection equation. First, we will use a first-order upwind approach to approximate the spatial derivative of $u^2$ at each grid point $x_i$, such that
\begin{equation}
	\Eval{\frac{\partial u^2}{\partial x}}{x_i}{} = \frac{(u_i^t)^2 - (u_{i-1}^t)^2}{\Delta x} + \mathcal{O}(\Delta x),
\end{equation}
where again the superscript $t$ denotes that we are using the solution at the current time. We will also use a first-order finite difference approach in time to approximate the temporal derivative as well. Hence,
\begin{equation}
	\Eval{\frac{\partial u}{\partial t}}{x_i}{} = \frac{u_i^{t+1} - u_{i}^t}{\Delta t} + \mathcal{O}(\Delta t).
\end{equation}

Now we simply insert our finite difference approximations of these two derivatives into the Burgers equation
\begin{equation}
	\frac{u_i^{t+1} - u_{i}^t}{\Delta t} +  \frac{(u_i^t)^2 - (u_{i-1}^t)^2}{\Delta x} + \mathcal{O}(\Delta x, \Delta t) = 0,
\end{equation}
and, just like for linear advection, we can now find an expression for the solution at any grid point at the next time step as
\begin{eqBox}
\begin{equation}
	u_i^{t+1} = u_{i}^t - \frac{\Delta t}{\Delta x} \left( \left(u_i^t \right)^2 - \left( u_{i-1}^t \right)^2 \right) + \mathcal{O}(\Delta x, \Delta t).
\end{equation}
\end{eqBox}
Looking at this expression we note it is remarkably similar to the expression for linear advection, which is not particularly surprising as the two initial partial differential equations are also quite similar. Again, if we know the solution at each grid point at some time $t$, we can approximate the solution at some time $t+\Delta t$ using this expression. Then, to approximate the solution at the time $t+2\Delta t$ we simply re-apply the expression. This allows us to approximate the behaviour of Burgers equation up to any desired final time, and the error of the approximation is expected to be first-order accurate in both space and time, with errors on the order of $\mathcal{O}(\Delta x, \Delta t)$.

\subsection{Linear Diffusion}
We recall that our third and final simplified system of equations was the linear diffusion equation
\begin{equation}
\frac{\partial u}{\partial t} - \beta \frac{\partial^2 u}{\partial x^2} = 0,
\end{equation}
which described how some conserved quantity, such as heat, diffuses into the surrounding fluid. To approximate this using the finite difference method we follow the exact same steps as for linear advection and Burgers equation. First, we will use our second-order accurate finite difference approximation for the second derivative as
\begin{equation}
	\Eval{\frac{\partial^2 u}{\partial x^2}}{x_i}{} = \frac{u_{i-1} - 2u_i + u_{i+1}}{\Delta x^2} + \mathcal{O}(\Delta x^2).
\end{equation}
We will also use a first-order finite difference approach in time to approximate the temporal derivative as well. Hence,
\begin{equation}
	\Eval{\frac{\partial u}{\partial t}}{x_i}{} = \frac{u_i^{t+1} - u_{i}^t}{\Delta t} + \mathcal{O}(\Delta t).
\end{equation}

Now we simply replace the exact derivatives in the linear diffusion equation with our finite difference approximations, such that
\begin{equation}
\frac{u_i^{t+1} - u_{i}^t}{\Delta t} - \beta \frac{u_{i-1} - 2u_i + u_{i+1}}{\Delta x^2} + \mathcal{O}(\Delta x^2,\Delta t) = 0.
\end{equation}
Again, by simply rearranging we can arrive at an expression for the solution at any grid point $i$ at the next time step as
\begin{eqBox}
\begin{equation}
u_i^{t+1}  = u_{i}^t + \frac{\beta \Delta t}{\Delta x^2} \left(u_{i-1} - 2u_i + u_{i+1}\right) + \mathcal{O}(\Delta x^2,\Delta t).
\end{equation}
\end{eqBox}
Again, this is very similar to the approximations for linear advection and Burgers equations. However, one important thing to note is that the error in space is of $\mathcal{O}(\Delta x^2)$ instead of $\mathcal{O}(\Delta x)$, meaning that as we refine the grid the solution will converge more quickly to the exact solution of the linear diffusion equation.

\chapter{Finite Volume Methods}
\section{Derivation}
\section{The Riemann Problem}
\section{Example Applications}
\subsection{Linear Advection}
\subsection{Burgers Equation}
\subsection{Linear Diffusion}

\chapter{Consistency, Stability, Convergence}
Now that we have derived a few different schemes for solving linear advection, Burgers equation, and linear diffusion, we should now ask ourselves whether they will give us accurate predictions, and are there any restrictions on when they can be used. We have already discussed the importance of order of accuracy, which governs the rate at which the error will converge to zero. Now we will introduce the concepts of consistency, stability, and convergence. If we can prove that our numerical scheme satisfies all three of these properties we can be confident that it is promising approach for the Euler or Navier-Stokes equations. In the current section we will explore these properties in the context of finite difference methods, but the exact same steps can be taken to check whether a finite volume method is suitable or not.

\section{Consistency}
A numerical scheme is consistent if it recovers the exact initial partial differential equation as the grid spacing and time step size are reduced. In other words, the truncation error of the scheme must go to zero in the limit $\Delta x \rightarrow 0$ and $\Delta t \rightarrow 0$. This is usually the case, and it is left as an exercise for the reader to check that our previous finite difference schemes for linear advection, Burgers equation, and linear diffusion satisfy this property. 

However, this is not always the case. To demonstrate this we consider the Dufort-Frankel scheme for linear diffusion
\begin{equation}
u_i^{t+1}  = u_{i}^{t-1} + \frac{2 \beta \Delta t}{\Delta x^2} \left(u_{i-1}^t - u_i^{t+1} - u_i^{t-1} + u_{i+1}^t\right) + \mathcal{O}(\Delta x^2,\Delta t^2,(\Delta t/\Delta x)^2).
\end{equation}
This differs from our example scheme for linear diffusion in that it uses central differences for the time-derivative, and when computing the second-derivative in space it uses the solution at the current grid point at the previous and current time steps. While this scheme may have some useful properties, we note that it has a peculiar term in the truncation error of $\mathcal{O}((\Delta t/\Delta x)^2)$. This can be obtained from a Taylor series expansion of the second derivative. This is concerning, as for any scheme to be consistent with the original partial differential equation we require the truncation error to go to zero. However, if we use a naive approach and simply refine $\Delta x$ and $\Delta t$ at the same rate, this term will not go to zero, and the scheme will not be consistent. For example, if we reduced both the grid spacing and time step size by a half this $\mathcal{O}((\Delta t/\Delta x)^2)$ will remain the same. Hence, in order for the Dufort-Frankel scheme to be consistent with our original partial differential equation we should refine the grid spacing {\it faster} than the time step size. For example, if we reduce $\Delta t$ by a half we should reduce $\Delta x$ by a factor of four. This does not mean the Dufort-Frankel scheme is bad, per-se, but it does mean that care needs to be taken when using it.

\section{Stability}
If we can demonstrate our schemes is consistent, then we know that in the limit $\Delta x \rightarrow 0$ and $\Delta t \rightarrow 0$ we recover the exact partial differential equation. However, this is impossible to achieve in practice as it would require an infinite number of grid points and time steps. When considering stability, we are concerned with whether our numerical scheme will provide {\it physical} solutions when both $\Delta x$ and $\Delta t$ are finite. Lets start by considering what we mean by a physical solution.

In order to advance our solution in time we start with some initial condition. Then, by inserting this initial condition into our scheme we approximate the solution at the next timestep $t+\Delta t$. Then, we insert this approximation back into our scheme to approximate the solution at time $t+2\Delta t$, and this process is repeated over an over again until we reach our final desired time. Hence, the way we advance our simulation in time is effectively a feedback loop, with the output of each time step being recycled back through the numerical scheme to get the solution at each consecutive time step. 

As an analogy we can consider what happens in other simple feedback loops, such as a microphone and speaker. When a performer sings into a microphone their voice is amplified and played bach through the speaker. We expect that this produces a physical replication of their voice, just at a louder volume for the audience. However, if the sound from the speaker is louder than the singers voice at the microphone it will get amplified, played through the speaker at a louder volume, and this cycle then repeats. This results in feedback noise, usually a high-pitched ringing that sounds nothing like the original performer, and usually happens when the performer moves to close to the speaker.

Since our numerical scheme is applied as a feedback loop, the exact same kind of thing can happen. If it amplifies our solution each time step, then the solution will continue to grow, eventually leading to non-physical values such as near-infinite density or pressure. This is colloquially referred to as the solution {\it blowing up}. In contrast, if the scheme damps our solution at each time step it will tend towards physical values, such as the background density or pressure. While this is perhaps not desirable in terms of accuracy, which will be explored later, it is a desirable property in that the solution remains stable and bounded between the initial condition and background state.

Proving stability for non-linear problems, such as Burgers equation, is a relatively daunting task. However, if we restrict ourselves to linear equations, such as linear advection or linear diffusion, then stability can be explored more readily. In order to do this we introduce {\it Von Neumann Analysis}, also commonly referred to as Fourier Analysis. We first assume that our solution $u(x,t)$ can be represented via a Fourier Series, such that
\begin{equation}
	u(x,t) = \sum_{m} b_m(t) e^{i \kappa_m x},
\end{equation}
where $b_m(t)$ is the Fourier coefficients that vary with time as the solution evolves, $\kappa_m$ is a wavenumber, and in this context $i = \sqrt{-1}$. This is the exact same as a conventional Fourer series, with the exception that the coefficients are a function of time describing the time evolution of the system of equations. Furthermore, we take
\begin{equation}
	\kappa_m = \frac{2 \pi m}{2 L}, \: m=0,1,2,\hdots,M,
\end{equation}
where $M$ is chosen based on the maximum wavenumber that can be represented on the grid based on its Nyquist criteria, and $L$ is the length of the domain. Hence, small values of $\kappa_m$ correspond to large waves, and large values of $\kappa_m$ correspond to very compact waves.

Following this approach, we are taking our solution and decomposing it into a number of different waves via a Fourier series. Now, since we have restricted ourselves to linear systems of equations, we can apply the property of superposition. This means that we can analyze each wave independently as a function of time, and the final solution is simply the superposition of all of these waves. This allows us to analyze the behaviour of our numerical scheme for each wavenumber independently, since they are not coupled. Hence, we can write the solution for one particular wave of the Fourier series as
\begin{equation}
	u_m(x,t) = b_m(t) e^{i \kappa_m x},
\end{equation}
where the complete solution is
\begin{equation}
	u(x,t) = \sum_{m} u_m(x,t).
\end{equation}
We will also assume that the time-dependence of the solution is also wavelike with a prescribed frequency in time such that
\begin{eqBox}
\begin{equation}
	u_m(x,t) = e^{at} e^{i \kappa_m x},
\end{equation}
\end{eqBox}
where $a$ is a complex number, referred to as the numerical frequency, that describes how the solution evolves in time. In order to justify this assumption we can consider the linear advection equation applied to an arbitrary wavenumber $\kappa_m$. We note that this wave will propagate at velocity $\alpha$ from left to right. Now if we consider some fixed point in space, denoted by $u_1{t}$, we note that the value of the solution in time will alose behave like a sine wave. Hence, at a fixed point in time, the solution has a wavelike structure in space and, at a fixed point in space, the solution has a wavelike structure in time. Hence, the dual wavelike structure taken for $u_m(x,t)$.

With the wavelike structure of the solution described, we can now explore how that wave will change with time. Of primary importance, at least in terms of stability, is to determine whether the wave will be amplified or damped as the solution evolves. Our objective in this section is to determine whether, and under what conditions, numerical schemes satisfy this stability condition. We note that the solution at time $t+\Delta t$ is simply
\begin{equation}
	u_m(x,t+\Delta t) = e^{a(t+\Delta t)} e^{i \kappa_m x},
\end{equation}
which can be re-written as
\begin{equation}
	u_m(x,t+\Delta t) = e^{at}e^{a\Delta t} e^{i \kappa_m x},
\end{equation}
and, in order for the magnitude of the solution to not be amplified at the next time step, we have the stability constraint
\begin{eqBox}
\begin{equation}
	|e^{a\Delta t}| \leq 1,
\end{equation}
\end{eqBox}
which describes a unit circle in the complex plane. Referred to as the amplification factor, we will next determine under what conditions our numerical schemes satisfy this stability constraint.

\subsection{Explicit Linear Advection}
For a methodological approach, we will break von Neumann down into a number of steps.

\subsubsection{Step 1: Choose the Discrete Scheme}
The first step in von Neumann analysis is to determine what scheme we are interested in analyzing. In this case we will consider our simple first-order scheme for linear advection
\begin{equation}
	\frac{u_i^{t+1} - u_{i}^t}{\Delta t} +  \alpha \frac{u_i^t - u_{i-1}^t}{\Delta x} = 0.
\end{equation}

\subsubsection{Step 2: Apply the Wavelike Solution}
Since we know the prescribed wave-like form of the solution, the grid spacing $\Delta x$, and the time step size $\Delta t$, we can write expressions for the solution for a particular wavenumber $\kappa_m$ at each grid point and time level as
\begin{equation}
	u_{i}^t = e^{at} e^{i \kappa_m x},
\end{equation}
\begin{equation}
	u_{i}^{t+1} = e^{a(t+\Delta t)} e^{i \kappa_m x},
\end{equation}
\begin{equation}
	u_{i-1}^t = e^{at} e^{i \kappa_m (x-\Delta x)}.
\end{equation}
Substituting these into our finite difference approximation yields
\begin{equation}
	\frac{e^{a(t+\Delta t)} e^{i \kappa_m x} - e^{at} e^{i \kappa_m x}}{\Delta t} +  \alpha \frac{e^{at} e^{i \kappa_m x} - e^{at} e^{i \kappa_m (x-\Delta x)}}{\Delta x} = 0.
\end{equation}

\subsubsection{Step 3: Solve for the Amplification Factor}
Noting that all of these terms has a common factor of $e^{at} e^{i \kappa_m x}$ we can simply divide through yielding
\begin{equation}
	\frac{e^{a\Delta t} - 1}{\Delta t} +  \alpha \frac{1 - e^{-i \kappa_m \Delta x}}{\Delta x} = 0.
\end{equation}
Rearranging yields
\begin{equation}
	e^{a\Delta t} = 1 - \sigma + \sigma e^{-i \kappa_m \Delta x},
\end{equation}
where
\begin{equation}
	\sigma = \frac{\alpha \Delta t}{\Delta x},
\end{equation}
is the {\it Courant-Friedrichs-Lewy} (CFL) number. Hence, the amplification factor of our first-order linear advection scheme is 
\begin{eqBox}
\begin{equation}
	|e^{a\Delta t}| = |1 - \sigma + \sigma e^{-i \kappa_m \Delta x}|,
\end{equation}
\end{eqBox}
and our scheme will be stable whenever this is contained within the unit circle in the complex plane. We note that this is a function of two parameters, specifically the wavenumber and the CFL number. Hence, we expect that the amount our solution gets amplified/damped each time step will depend on these two parameters.

\subsubsection{Step 4: Determine the Stability Conditions}
Based on these results, we can conclude that the first-order finite difference scheme for linear advection is stable whenever
\begin{eqBox}
\begin{equation}
	0 \leq \sigma \leq 1.
\end{equation}
\end{eqBox}
That is, it is only stable for CFL numbers less than one. Hence, for a given grid spacing and advection velocity, there is a limit on how large the time step can be. This clearly has implications in terms of computational cost, as the smaller the time step is the more steps must be taken to reach a desired final solution time. Schemes of this type are referred to as being {\it conditionally stable}.

\subsection{Implicit Linear Advection}
In the last section we saw that our first-order finite difference scheme has a stability limit. In this section we will explore a slightly different first-order scheme for linear advection.

\subsubsection{Step 1: Choose the Discrete Scheme}
In this case we use the following finite difference approximation of the linear advection equation
\begin{equation}
	\frac{u_i^{t+1} - u_{i}^t}{\Delta t} +  \alpha \frac{u_i^{t+1} - u_{i-1}^{t+1}}{\Delta x} = 0,
\end{equation}
noting that it is simple the original scheme, but we evaluate the spatial derivative at the next time step rather than the current time step.

\subsubsection{Step 2: Apply the Wavelike Solution}
Our expressions for wavelike solutions at each point are
\begin{equation}
	u_{i}^t = e^{at} e^{i \kappa_m x},
\end{equation}
\begin{equation}
	u_{i}^{t+1} = e^{a(t + \Delta t)} e^{i \kappa_m x},
\end{equation}
\begin{equation}
	u_{i-1}^{t+1} = e^{a(t + \Delta t)} e^{i \kappa_m (x - \Delta x)}.
\end{equation}
Substituting these into our discrete scheme yields
\begin{equation}
	\frac{e^{a(t + \Delta t)} e^{i \kappa_m x} - e^{at} e^{i \kappa_m x}}{\Delta t} +  \alpha \frac{e^{a(t + \Delta t)} e^{i \kappa_m x} - e^{a(t + \Delta t)} e^{i \kappa_m (x - \Delta x)}}{\Delta x} = 0.
\end{equation}

\subsubsection{Step 3: Solve for the Amplification Factor}
Again we note a common factor of $e^{at} e^{i \kappa_m x}$ in all terms, allowing us to divide through yielding
\begin{equation}
	\frac{e^{a\Delta t} - 1}{\Delta t} +  \alpha \frac{e^{a\Delta t} - e^{a\Delta t} e^{-i \kappa_m \Delta x}}{\Delta x} = 0.
\end{equation}
Rearranging yields
\begin{equation}
	e^{a\Delta t} = \frac{1}{1 + \sigma \left( 1 - e^{-i \kappa_m \Delta x} \right)},
\end{equation}
where $\sigma$ is again the CFL number. Hence, the amplification factor is
\begin{eqBox}
\begin{equation}
	|e^{a\Delta t}| = \left| \frac{1}{1 + \sigma \left( 1 - e^{-i \kappa_m \Delta x} \right)} \right|.
\end{equation}
\end{eqBox}

\subsubsection{Step 4: Determine the Stability Conditions}
Based on these results, we can conclude that this finite difference scheme for the linear advection equation is stable provided
\begin{eqBox}
\begin{equation}
	0 \leq \sigma \leq \infty.
\end{equation}
\end{eqBox}
Hence, we are able to take arbitrarily large time steps and maintain stability using this approach. Schemes of this type are referred to as being {\it unconditionally stable}.

\begin{remark}
Although this scheme is unconditionally stable, making at appealing since it allows for arbitrarily large time-steps, it also becomes more difficult/expensive for each time-step. This will be explored in the forthcoming time stepping section.
\end{remark}

\subsection{Explicit Linear Diffusion}
Similar to the linear advection equation, we can also use von Neumann analysis to analyze the stability of the linear diffusion equation. 

\subsubsection{Step 1: Choose the Discrete Scheme}
We will start with the example scheme we derived in the finite difference section
\begin{equation}
  \frac{u_i^{t+1} - u_{i}^t}{\Delta t} - \beta \frac{u_{i-1}^t - 2u_i^t + u_{i+1}^t}{\Delta x^2} = 0.
\end{equation}

\subsubsection{Step 2: Apply the Wavelike Solution}
Our expressions for wavelike solutions at each point are
\begin{equation}
	u_{i}^t = e^{at} e^{i \kappa_m x},
\end{equation}
\begin{equation}
	u_{i}^{t+1} = e^{a(t + \Delta t)} e^{i \kappa_m x},
\end{equation}
\begin{equation}
	u_{i-1}^{t} = e^{at} e^{i \kappa_m (x - \Delta x)},
\end{equation}
\begin{equation}
	u_{i+1}^{t} = e^{at} e^{i \kappa_m (x + \Delta x)}.
\end{equation}
Substituting these into our discrete scheme yields
\begin{equation}
  \frac{e^{a(t + \Delta t)} e^{i \kappa_m x} - e^{at} e^{i \kappa_m x}}{\Delta t} -  \beta \frac{e^{at} e^{i \kappa_m (x - \Delta x)} - 2e^{at} e^{i \kappa_m x} + e^{at} e^{i \kappa_m (x + \Delta x)}}{\Delta x ^2} = 0.
\end{equation}

\subsubsection{Step 3: Solve for the Amplification Factor}
Again we note a common factor of $e^{at} e^{i \kappa_m x}$ in all terms, allowing us to divide through yielding
\begin{equation}
  \frac{e^{a\Delta t} - 1}{\Delta t} -  \beta \frac{e^{-i \kappa_m \Delta x} - 2 + e^{i \kappa_m \Delta x}}{\Delta x ^2} = 0.
\end{equation}
Rearranging yields
\begin{equation}
	e^{a\Delta t} = 1 + r \left( e^{-i \kappa_m \Delta x} - 2 + e^{i \kappa_m \Delta x} \right), 
\end{equation}
where
\begin{equation}
	r = \frac{\beta \Delta t}{\Delta x^2},
\end{equation}
is similar to the CFL number but for diffusion rather than advection. Finally, the amplification factor is
\begin{eqBox}
\begin{equation}
	|e^{a\Delta t}| = \left| 1 + r \left( e^{-i \kappa_m \Delta x} - 2 + e^{i \kappa_m \Delta x} \right) \right|.
\end{equation}
\end{eqBox}

\subsubsection{Step 4: Determine the Stability Conditions}
Based on this amplification factor, we demonstrate graphically that this scheme will be stable provided
\begin{eqBox}
\begin{equation}
	0 \leq r \leq \frac{1}{2}.
\end{equation}
\end{eqBox}
We note that this scheme is {\it conditionally stable}, similar to the first linear advection scheme we considered. This means if the grid is refined then the time step size must be reduced accordingly to maintain stability. However, we note that the factor of $\Delta x^2$ in $r$ will require the time step size to be reduced with the square of the grid spacing, which can become expensive on finer meshes.

\begin{remark}
The Navier-Stokes equations have both advective and diffusive terms. Hence, the time step is usually limited by the stricter of the stability constraints of the advective and diffusive schemes that are being used.
\end{remark}

\subsection{Implicit Linear Diffusion}
Similar to the modified linear advection scheme, we can also modify our initial linear diffusion scheme by evaluating the spatial operator at the future unknown solution time.

\subsubsection{Step 1: Choose the Discrete Scheme}
This yields the following scheme
\begin{equation}
  \frac{u_i^{t+1} - u_{i}^t}{\Delta t} - \beta \frac{u_{i-1}^{t+1} - 2u_i^{t+1} + u_{i+1}^{t+1}}{\Delta x^2} = 0.
\end{equation}

\subsubsection{Step 2: Apply the Wavelike Solution}
Our expressions for wavelike solutions at each point are
\begin{equation}
	u_{i}^t = e^{at} e^{i \kappa_m x},
\end{equation}
\begin{equation}
	u_{i}^{t+1} = e^{a(t + \Delta t)} e^{i \kappa_m x},
\end{equation}
\begin{equation}
	u_{i-1}^{t+1} = e^{a(t + \Delta t)} e^{i \kappa_m (x - \Delta x)},
\end{equation}
\begin{equation}
	u_{i+1}^{t+1} = e^{a(t + \Delta t)} e^{i \kappa_m (x + \Delta x)}.
\end{equation}
Substituting these into our discrete scheme yields
\begin{equation}
  \frac{e^{a(t + \Delta t)} e^{i \kappa_m x} - e^{at} e^{i \kappa_m x}}{\Delta t} - \beta \frac{e^{a(t + \Delta t)} e^{i \kappa_m (x - \Delta x)} - 2e^{a(t + \Delta t)} e^{i \kappa_m x} + e^{a(t + \Delta t)} e^{i \kappa_m (x + \Delta x)}}{\Delta x^2} = 0.
\end{equation}

\subsubsection{Step 3: Solve for the Amplification Factor}
Again we note a common factor of $e^{at} e^{i \kappa_m x}$ in all terms, allowing us to divide through yielding
\begin{equation}
  \frac{e^{a\Delta t} - 1}{\Delta t} - \beta \frac{e^{a\Delta t} e^{-i \kappa_m \Delta x} - 2e^{a\Delta t} + e^{a\Delta t} e^{i \kappa_m \Delta x}}{\Delta x^2} = 0.
\end{equation}
Rearranging yields
\begin{equation}
e^{a\Delta t}	= \frac{1}{\left[ 1 - r \left(e^{-i \kappa_m \Delta x} - 2 + e^{i \kappa_m \Delta x} \right)  \right]},
\end{equation}
where $r$ is the same as the original linear diffusion scheme. Finally, the amplification factor is
\begin{eqBox}
\begin{equation}
	|e^{a\Delta t}|	= \left| \frac{1}{\left[ 1 - r \left(e^{-i \kappa_m \Delta x} - 2 + e^{i \kappa_m \Delta x} \right)  \right]} \right|.
\end{equation}
\end{eqBox}

\subsubsection{Step 4: Determine the Stability Conditions}
Based on the form of the amplification factor, we can conclude that this scheme will be stable for
\begin{eqBox}
\begin{equation}
	0 \leq r \leq \infty.
\end{equation}
\end{eqBox}
Hence, this linear diffusion scheme is {\it unconditionally stable}.

\section{Convergence}
If a scheme is consistent, recovering the exact partial differential equation in the limit $\Delta x \rightarrow 0$ and $\Delta t \rightarrow 0$, and stable, in that the approximate solution does not grow unbounded with time, then Lax's equivalence theorem can be applied. In summary, the theorem states that when given a properly-posed initial value problem, and a numerical scheme that is consistent, stability is the necessary and sufficient condition for convergence. In other words, if we can show that our numerical schemes is consistent and stable, then we can be sure that it will {\it converge} to the true solution of the partial differential equation in the limit $\Delta x \rightarrow 0$ and $\Delta t \rightarrow 0$.

\chapter{Time-Stepping}
In the previous sections we have always approximated the time derivative using first-order finite differences, assuming
\begin{equation}
	\frac{\partial u}{\partial t} = \frac{u_i^{t+1} - u_{i}^t}{\Delta t} + \mathcal{O}(\Delta t).
\end{equation}
In the von Neumann analysis section we demonstrated that two linear advection schemes and two linear diffusion schemes are either conditionally or unconditionally stable using this approach. However, if we try to use this finite difference approach to get higher-order accurate schemes in time, such as a central in time approach
\begin{equation}
	\frac{\partial u}{\partial t} = \frac{u_i^{t+1} - u_{i}^{t-1}}{2 \Delta t} + \mathcal{O}(\Delta t^2),
\end{equation}
we find it is almost always {\it unstable}. Hence, getting higher than first-order accurate solutions in time requires something other than finite differences. Furthermore, for both of the unconditionally stable schemes in the von Neumann section, we have not yet discussed how to actually advance them in time, and how to get higher-order accuracy in time, while maintaining this unconditional stability. Hence, in this section we will discuss more appropriate discretizations for the time derivative.

\section{Explicit}
\section{Implicit}

\chapter{Spectral Properties}
\section{Dissipation}
\section{Dispersion}

\chapter{Modified Equation Analysis}

\chapter{Iterative Methods}
In this implicit time stepping section we reduced each time step to the solution of a linear system of $N$ equations of the form
\begin{equation}
	A \vec{x} = \vec{b},
\end{equation}
when solving either linear advection or linear diffusion, where $N$ is the number of grid points in the domain. The extra cost of solving this linear system was introduced in order to obtain unconditionally stable schemes, allowing very large time steps to be taken. However, this will only be faster than an explicit approach if the linear system of equations can be solved efficiently. Hence, this section is dedicated to exploring different methods of solving linear systems of equations.

\section{Gaussian Elimination}
Perhaps the most straightforward method for solving a linear system of equations is Guassian elimination. This requires us to invert $A$ and obtain
\begin{equation}
	 \vec{x} = A^{-1} \vec{b}.
\end{equation}
Gaussian elimination is the workhorse of most undergraduate linear algebra courses, and is a straightforward extension of the substitution approach for solving systems of equations taught in highschool. This requires inverting $A$ or an LU decomposition of $A$ and $\vec{b}$ to be formed, and then backsubstitution to solve for the unknown components of $\vec{x}$.

While Gaussian elimination is well known, its computational cost scales like $\mathcal{O}(N^3)$ where $N$ is the number of unknowns. Hence, as the system of equations, and in our case number of grid points, grows, the computational cost of Gaussian elimination grows cubicly. Any undergraduate student should be familiar with trying to solve even $3 \times 3$ linear systems in an exam, let alone $4 \times 4$, or $5 \times 5$. With each additional equation, doing this by hand takes significantly more time. Hence, in the context of CFD, where thousands or millions of equations is common practice, Gaussian elimination is rarely used.

\section{Jacobi Iteration}
Since the formation of $A^{-1}$ is prohibitively expensive for large systems of equations, we will try to find an approximate solution $\vec{x}$, without actually forming $A^{-1}$. This can be done iteratively, and the first approach we will introduce is Jacobi iteration. In Jacobi iteration we start by splitting $A$ into $L$, $D$, and $U$, and are its strictly lower-triangular, diagonal, and strictly upper-triangular components. This allows us to write our linear system as
\begin{equation}
	(L+D+U) \vec{x} = \vec{b},
\end{equation}
where
\begin{equation}
	A = 
	\begin{bmatrix}
	    a_{11} & a_{12} & a_{13} & \dots  & a_{1n} \\
	    a_{21} & a_{22} & a_{23} & \dots  & a_{2n} \\
			a_{31} & a_{32} & a_{33} & \dots  & a_{3n} \\
	    \vdots & \vdots & \vdots & \ddots & \vdots \\
	    a_{n1} & a_{n2} & a_{n3} & \dots  & a_{nn}
	\end{bmatrix},
\end{equation}
\begin{equation}
	L = 
	\begin{bmatrix}
	    0 & 0 & 0 & \dots  & 0 \\
	    a_{21} & 0 & 0 & \dots  & 0 \\
			a_{31} & a_{32} & 0 & \dots  & 0 \\
	    \vdots & \vdots & \vdots & \ddots & \vdots \\
	    a_{n1} & a_{n2} & a_{n3} & \dots  & 0
	\end{bmatrix},
\end{equation}
\begin{equation}
	D = 
	\begin{bmatrix}
	    a_{11} & 0 & 0 & \dots  & 0 \\
	    0 & a_{22} & 0 & \dots  & 0 \\
			0 & 0 & a_{33} & \dots  & 0 \\
	    \vdots & \vdots & \vdots & \ddots & \vdots \\
	    0 & 0 & 0 & \dots  & a_{nn}
	\end{bmatrix},
\end{equation}
\begin{equation}
	U = 
	\begin{bmatrix}
	    0 & a_{12} & a_{13} & \dots  & a_{1n} \\
	    0 & 0 & a_{23} & \dots  & a_{2n} \\
			0 & 0 & 0 & \dots  & a_{3n} \\
	    \vdots & \vdots & \vdots & \ddots & \vdots \\
	    0 & 0 & 0 & \dots  & 0
	\end{bmatrix}.
\end{equation}
Now if $\vec{x}$ is a solution to the linear system then the following is valid
\begin{equation}
	L\vec{x} + D\vec{x} + U\vec{x} = \vec{b}.
\end{equation}
We can now rearrange this expression
\begin{equation}
	D\vec{x} = \vec{b} - (L+U)\vec{x},
\end{equation}
and by inverting $D$ we get
\begin{equation}
	\vec{x} = D^{-1}\left[ \vec{b} - (L+U)\vec{x} \right].
	\label{eqn:f928bs7g}
\end{equation}
At this point it is important to now that while inverting $A$ is very expensive, inverting just its diagonal values in $D$ is trivial.

We note that if $\vec{x}$ is a solution to the linear system of equations, then Equation \ref{eqn:f928bs7g} will be satisfied exactly, but if we already had the exact solution there would be no point in this exercise. However, if we look again at Equation \ref{eqn:f928bs7g} it has an interesting form. We note that if we insert some guess for $\vec{x}$ into the right hand side, we get out a modified $\vec{x}$ on the left hand side. This allows us to define Jacobi iteration as
\begin{eqBox}
\begin{equation}
	\vec{x}_{n+1} = D^{-1}\left[ \vec{b} - (L+U)\vec{x}_n \right],
	\label{eqn:3jc810fg}
\end{equation}
\end{eqBox}
where $\vec{x}_n$ is an approximate solution, and $\vec{x}_{n+1}$ is an updated approximation. Then we can simply pass $\vec{x}_{n+1}$ back through this formulation to obtain $\vec{x}_{n+2}$, and so on, iterating to a final solution when the value of $\vec{x}$ converges. Each step of Jacobi iteration requires $\mathcal{O}(N^2)$ operations, hence, we can expect Jacobi to outperform Gaussian elimination, as long as we need fewer iterations than there are rows in the system of equations. Furthermore, in CFD it is usually sufficient to converge $\vec{x}$ to some finite level of precision.

To demonstrate the utility of Jacobi iteration we will use a simple example of a $3 \times 3$ system of equations. Starting with 
\begin{equation}
	A = 
	\begin{bmatrix}
	    10 & 2 & 4 \\
	    6 & 8 & 4 \\
			2 & 3 & 9
	\end{bmatrix},
\end{equation}
and
\begin{equation}
	\vec{b} = \begin{bmatrix}
	    1 \\
	    2 \\
			3
	\end{bmatrix},
\end{equation}
we obtain the following for $L$, $D$, and $U$
\begin{equation}
	L = 
	\begin{bmatrix}
	    0 & 0 & 0 \\
	    6 & 0 & 0 \\
			2 & 3 & 0
	\end{bmatrix},
\end{equation}
\begin{equation}
	D = 
	\begin{bmatrix}
	    10 & 0 & 0 \\
	    0 & 8 & 0 \\
			0 & 0 & 9
	\end{bmatrix},
\end{equation}
\begin{equation}
	U = 
	\begin{bmatrix}
	    0 & 2 & 4 \\
	    0 & 0 & 4 \\
			0 & 0 & 0
	\end{bmatrix}.
\end{equation}
Furthermore, we can easily invert $D$ to get
\begin{equation}
	D^{-1} = 
	\begin{bmatrix}
	    \frac{1}{10} & 0 & 0 \\
	    0 & \frac{1}{8} & 0 \\
			0 & 0 & \frac{1}{9}
	\end{bmatrix}.
\end{equation}
Now we can start with some initial guess, lets say
\begin{equation}
	\vec{x_0} = \begin{bmatrix}
	    0 \\
	    0 \\
			0
	\end{bmatrix}.
\end{equation}
Convergence can usually be accelerated by using a good initial guess, for example, that the solution at the next time step in our CFD simulation is the same as the current solution.

Now with our initial Guess and all terms from the right hand side of Equation \ref{eqn:3jc810fg} defined, we can compute
\begin{equation}
	\vec{x}_{1} = D^{-1}\left[ \vec{b} - (L+U)\vec{x}_0 \right] = \begin{bmatrix}
	    0.1000 \\
	    0.2500 \\
			0.3333
	\end{bmatrix}.
\end{equation}
Repeating this iteration a few times yields
\begin{equation}
	\vec{x}_{2} = D^{-1}\left[ \vec{b} - (L+U)\vec{x}_1 \right] = \begin{bmatrix}
	    -0.0833 \\
	    0.0083 \\
			0.2277
	\end{bmatrix},
\end{equation}
\begin{equation}
	\vec{x}_{3} = D^{-1}\left[ \vec{b} - (L+U)\vec{x}_2 \right] = \begin{bmatrix}
	    0.0072 \\
	    0.1986 \\
			0.3490
	\end{bmatrix},
\end{equation}
and after 15 iterations we obtain
\begin{equation}
	\vec{x}_{15} = D^{-1}\left[ \vec{b} - (L+U)\vec{x}_14 \right] = \begin{bmatrix}
	    -0.0465 \\
	    0.1356 \\
			0.2984
	\end{bmatrix},
\end{equation}
which is very close to the exact solution
\begin{equation}
	\vec{x} = \begin{bmatrix}
	    -0.0465 \\
	    0.1357 \\
			0.2984
	\end{bmatrix}.
\end{equation}
Hence, after only 15 iterations Jacobi was able to converge to approximately four digits, without having to ever directly invert the $A$ matrix.

\section{Gauss Seidel Iteration}
With the idea of Jacobi iteration outlined above, we may wonder whether similar more efficient approaches exist. The first of these, with a very similar derivation to Jacobi, is the Gauss-Seidel method. Starting from our split $A$ matrix in the Jacobi section
\begin{equation}
	(L+D+U) \vec{x} = \vec{b},
\end{equation}
we rearrange it as
\begin{equation}
	(L + D)\vec{x} = \vec{b} - U\vec{x}.
\end{equation}
If we consider the $(L+D)$ matrix we note that it is lower-triangular. While not as trivial to invert as $D$ in Jacobi iteration, inverting $(L+D)$ can be done quickly via back-substitution. Hence, we can write
\begin{equation}
	\vec{x} = (L + D)^{-1}\left[ \vec{b} - U\vec{x} \right].
\end{equation}
Similar to Jacobi, we notice that if we insert an estimate for $\vec{x}$ on the right hand side, we obtain an updated estimate on the left hand side. Hence, we define Gauss-Seidel iteration as
\begin{eqBox}
\begin{equation}
	\vec{x}_{n+1} = (L + D)^{-1}\left[ \vec{b} - U\vec{x}_n \right].
\end{equation}
\end{eqBox}
Usually the additional cost of computing $(L + D)^{-1}$, relative to simply computing $D^{-1}$ with Jacobi, significantly reduces the number of iterations and total computational cost. 

\section{Successive Over-Relaxation}
When using the Gauss-Seidel approach, we often observe that the iterated solutions converge gradually to the exact solution for $\vec{x}$. That is, each iteration takes a step towards the exact solution in a somewhat uniform manner. We can exploit this by simply extrapolating each iteration to move a bit further towards the exact solution, which is known as Successive Over-Relaxation (SOR). This is usually done to accelerate convergence, but as will be discussed later, it can be used to also stabilize convergence.

With SOR we simply apply a Gauss Seidel iteration to our current approximation of the solution, $\vec{x}_n$. We store the results of this Gauss-Seidel iteration temporarily as $\tilde{\vec{x}}_{n+1}$. Then, using the SOR approach, we linearly project our updated solution using our current and previous approximations, that is
\begin{eqBox}
\begin{equation}
	\vec{x}_{n+1} = \omega \tilde{\vec{x}}_{n+1} + (1-\omega) \vec{x}_n.
\end{equation}
\end{eqBox}
where $\omega$ is referred to as the the relaxation factor. We note that when $\omega=1$ we recover the original Gauss Seidel approach. When $\omega>1$ we say it is over-relaxed, usually accelerating convergence but also potentially causing the iterations to diverge. When $\omega<1$ we say it is under-relaxed, which usually improves stability, but reduces the rate of convergence. In general, a suitable value for the relaxation factor is within the range
\begin{equation}
	0 \leq \omega \leq 2,
\end{equation}
and values of $\omega > 2$ will diverge. Typically, $\omega$ is chosen to be as large as possible, without causing the iterations to diverge.

\section{Assessing Convergence}
In typical CFD applications, the user provides a desired convergence tolerance for solving the linear system of equations. To assess this, we introduce the concept of a residual. In the case of an exact solution to a linear system we would expect
\begin{equation}
	A \vec{x} - \vec{b} = 0.
\end{equation}
However, since our solution at any given iteration is only an approximate solution then
\begin{equation}
	A \vec{x}_n - \vec{b} = \vec{r},
\end{equation}
where $\vec{r}$ is referred to as the residual and measures how well our current approximate solution satisfies the system of equations being solved. While $\vec{r}$ gives detailed information about how well each equation in the entire system is being solved, it is typically more useful to provide the user with a norm, such as $\| \vec{r} \|_2$ or $\| \vec{r} \|_\infty$, giving a single measure of convergence. Then, iterations are continued until this residual norm converges to within the desired tolerance. It is also important to note that if the convergence tolerance is too high it can results in a loss of accuracy or stability in the CFD simulation. Hence, particular care must be chosen in how to select the desired convergence tolerance.

\section{Multigrid}